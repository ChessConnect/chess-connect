\documentclass{article}

\usepackage{booktabs}
\usepackage{tabularx}
\usepackage{hyperref}
\usepackage{amsmath, mathtools}
\usepackage{amsfonts}
\usepackage{amssymb}
\usepackage{graphicx}
\usepackage{colortbl}
\usepackage{xr}
\usepackage{hyperref}
\usepackage{longtable}
\usepackage{xfrac}
\usepackage{tabularx}
\usepackage{booktabs}
\usepackage{graphicx}
\usepackage{float}
\usepackage{siunitx}
\usepackage{caption}
\usepackage{pdflscape}
\usepackage{afterpage}
\usepackage{fullpage}
\usepackage{titlesec}
\usepackage[shortlabels]{enumitem}
\usepackage[round]{natbib}
\usepackage{array}

\hypersetup{
    colorlinks=true,       % false: boxed links; true: colored links
    linkcolor=red,          % color of internal links (change box color with linkbordercolor)
    citecolor=green,        % color of links to bibliography
    filecolor=magenta,      % color of file links
    urlcolor=cyan           % color of external links
}

\title{Hazard Analysis\\\progname}

\author{\authname}

\date{}

%% Comments

\usepackage{color}

\newif\ifcomments\commentstrue %displays comments
%\newif\ifcomments\commentsfalse %so that comments do not display

\ifcomments
\newcommand{\authornote}[3]{\textcolor{#1}{[#3 ---#2]}}
\newcommand{\todo}[1]{\textcolor{red}{[TODO: #1]}}
\else
\newcommand{\authornote}[3]{}
\newcommand{\todo}[1]{}
\fi

\newcommand{\wss}[1]{\authornote{blue}{SS}{#1}} 
\newcommand{\plt}[1]{\authornote{magenta}{TPLT}{#1}} %For explanation of the template
\newcommand{\an}[1]{\authornote{cyan}{Author}{#1}}

%% Common Parts

\newcommand{\progname}{Chess Connect} % PUT YOUR PROGRAM NAME HERE
\newcommand{\authname}{Team \#4,
\\ Alexander Van Kralingen
\\ Arshdeep Aujla
\\ Jonathan Cels
\\ Joshua Chapman
\\ Rupinder Nagra} % AUTHOR NAMES without MacIDs 

\usepackage{hyperref}
    \hypersetup{colorlinks=true, linkcolor=blue, citecolor=blue, filecolor=blue,
                urlcolor=blue, unicode=false}
    \urlstyle{same}
                                


\begin{document}

\maketitle
\thispagestyle{empty}

~\newpage

\pagenumbering{roman}

\begin{table}[hp]
\caption{Revision History} \label{TblRevisionHistory}
\begin{tabularx}{\textwidth}{llX}
\toprule
\textbf{Date} & \textbf{Developer(s)} & \textbf{Change}\\
\midrule
Date1 & Name(s) & Description of changes\\
Date2 & Name(s) & Description of changes\\
... & ... & ...\\
\bottomrule
\end{tabularx}
\end{table}

~\newpage

\tableofcontents

~\newpage

\pagenumbering{arabic}

\wss{You are free to modify this template.}

\section{Introduction}

\wss{You can include your definition of what a hazard is here.}

\section{Scope and Purpose of Hazard Analysis}

\section{System Boundaries and Components}

\section{Critical Assumptions}

\wss{These assumptions that are made about the software or system.  You should
minimize the number of assumptions that remove potential hazards.  For instance,
you could assume a part will never fail, but it is generally better to include
this potential failure mode.}

\section{Failure Mode and Effect Analysis}
{The following table (Table 2) is a breakdown of the failure modes and and effects analysis (FMEA) for the Chess Connect system.}
\begin{table}[ht]
    \centering
        \setlength{\leftmargini}{0.4cm}
        \begin{tabular}{| >{\centering\arraybackslash}m{2.5cm} | 
          >{\centering\arraybackslash}m{2cm} | 
          >{\centering\arraybackslash}m{3cm} |
          >{\centering\arraybackslash}m{2cm} |
          >{\centering\arraybackslash}m{3cm} |
          >{\centering\arraybackslash}m{1.5cm} |
          >{\centering\arraybackslash}m{1cm} |}
        \hline
        \rowcolor[gray]{0.9}
        Component & Failure & Causes & Detection & Recommended Action & Probability of Occurence & Ref. \\
        \hline
        Web Application & Loss of Internet connection
        & \begin{enumerate}[label=(\alph*)]
            \item Internet outage 
            \item Loss of power
            \item Internet time-out 
        \end{enumerate} & Ping the Internet and wait for the response & Alert the user to check Internet connection & 0.3 & TBD \\
        \hline
        Microcontroller & Bad inputs &
        \begin{enumerate}[label=(\alph*)]
            \item If a player knocks down multiple pieces in their turn 
            \item Loss of power
            \item Faulty components and/or connections 
        \end{enumerate} & Monitoring inputs & Prompt the user to return the system to previous state and redo the turn & 0.4 & TBD \\ 
        \hline
        Microcontroller & Loss of Bluetooth connection & \begin{enumerate}[label=(\alph*)]
            \item Distance between microcontroller and host is too large
            \item Physical barriers between microcontroller and host
            \item Failed to initialise connection
        \end{enumerate} & Continuously monitor Bluetooth connection & Prompt user to re-establish connection before continuing & 0.2 & TBD \\
        \hline
        Hall Sensor & Bad inputs & \begin{enumerate}[label=(\alph*)]
            \item Sensitivity loss over a period of time
            \item Interference from external magnetic objects
            \item Distance between sensor and object too large
        \end{enumerate} & Monitoring Hall sensor inputs & \begin{enumerate}[label=(\alph*)]
            \item Prompt the user to clear area of obstacles from the board
            \item The sensor should be replaced after the recommended use time
        \end{enumerate} & 0.1 & TBD \\
        \hline
        \end{tabular}
    \caption{Failure Mode and Effects Analysis}
    \end{table}

\section{Safety and Security Requirements}

\wss{Newly discovered requirements.  These should also be added to the SRS.  (A
rationale design process how and why to fake it.)}

\section{Roadmap}

\wss{Which safety requirements will be implemented as part of the capstone timeline?
Which requirements will be implemented in the future?}

\end{document}
