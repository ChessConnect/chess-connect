\documentclass[12pt, titlepage]{article}

\usepackage{booktabs}
\usepackage{tabularx}
\usepackage{hyperref}
\usepackage[shortlabels]{enumitem}
\hypersetup{
    colorlinks,
    citecolor=blue,
    filecolor=black,
    linkcolor=red,
    urlcolor=blue
}
\usepackage[round]{natbib}

%% Comments

\usepackage{color}

\newif\ifcomments\commentstrue %displays comments
%\newif\ifcomments\commentsfalse %so that comments do not display

\ifcomments
\newcommand{\authornote}[3]{\textcolor{#1}{[#3 ---#2]}}
\newcommand{\todo}[1]{\textcolor{red}{[TODO: #1]}}
\else
\newcommand{\authornote}[3]{}
\newcommand{\todo}[1]{}
\fi

\newcommand{\wss}[1]{\authornote{blue}{SS}{#1}} 
\newcommand{\plt}[1]{\authornote{magenta}{TPLT}{#1}} %For explanation of the template
\newcommand{\an}[1]{\authornote{cyan}{Author}{#1}}

%% Common Parts

\newcommand{\progname}{Chess Connect} % PUT YOUR PROGRAM NAME HERE
\newcommand{\authname}{Team \#4,
\\ Alexander Van Kralingen
\\ Arshdeep Aujla
\\ Jonathan Cels
\\ Joshua Chapman
\\ Rupinder Nagra} % AUTHOR NAMES without MacIDs 

\usepackage{hyperref}
    \hypersetup{colorlinks=true, linkcolor=blue, citecolor=blue, filecolor=blue,
                urlcolor=blue, unicode=false}
    \urlstyle{same}
                                


\begin{document}

\title{Project Title: System Verification and Validation Plan for \progname{}} 
\author{\authname}
\date{\today}
	
\maketitle

\pagenumbering{roman}

\section{Revision History}

\begin{table}[hp]
\caption{Revision History} \label{TblRevisionHistory}
\begin{tabularx}{\textwidth}{llX}
\toprule
\textbf{Date} & \textbf{Developer(s)} & \textbf{Change}\\
\midrule
2022-10-31 & Jonathan Cels & NFR Testing\\
2022-10-31 & Arshdeep Aujla & Added section 3\\
2022-11-02 & Joshua Chapman & Section 4.1, 4.2, 4.3, 4.5\\
2022-11-02 & Alexander Van Kralingen & Completed Section 4.4, 4.6, 4.7\\
\bottomrule
\end{tabularx}
\end{table}

\newpage

\tableofcontents

\listoftables
\wss{Remove this section if it isn't needed}

\listoffigures
\wss{Remove this section if it isn't needed}

\newpage

\section{Symbols, Abbreviations and Acronyms}

\renewcommand{\arraystretch}{1.2}
\begin{tabular}{l l} 
  \toprule		
  \textbf{symbol} & \textbf{description}\\
  \midrule 
  T & Test\\
  \bottomrule
\end{tabular}\\

\wss{symbols, abbreviations or acronyms --- you can simply reference the SRS
  \citep{SRS} tables, if appropriate}

\wss{Remove this section if it isn't needed}

\newpage

\pagenumbering{arabic}

This document ... \wss{provide an introductory blurb and roadmap of the
  Verification and Validation plan}

\section{General Information}

\subsection{Summary}
The project name is Chess Connect. It is comprised of software and hardware components. The hardware will consist of a reactive chess set 
connected to a microcontroller. The microcontroller will relay information on the chess board in the form of LEDs of the possible moves the user can make.
The software component of this project will consist of a web application that will reflect all of the chess piece's location on the physical board.

\subsection{Objectives}
The following objectives are the qualities that are the most important for the project.
\begin{itemize}
  \item The hardware should reflect relevent information on the LEDs on the chess board
  \item The software component should reflect the physical chess board in near-real time
  \item The movement of the chess pieces should be recorded by the hardware
\end{itemize}

\subsection{Relevant Documentation}
The following documents are relevent to this project.
\begin{itemize}
  \item SRS
  \item Hazard Analysis
  \item Requirements Document
  \item Design Document
  \item VnV Report
  \item MIS
  \item MG
\end{itemize}

\section{Plan}
This section discusses the general plans for hardware and software testing. The 
responsibilities of members are assigned and requirements verifications are discussed. 
Design and implementation verification include high-level plans to execute hardware 
and software systems. Automated testing techniques and tools are outlined. 
A detailed software validation plan discusses the external tools required for 
completeness. 
%Introduce this section.   You can provide a roadmap of the sections to come.

\subsection{Verification and Validation Team}

\begin{tabular}{ |p{4.5cm}||p{9cm}|  }
 \hline
 Alexander Van Kralingen & Web application connection tests\\
 & GitHub integration testing\\
 & Microcontroller code testing\\
 \hline
 Arshdeep Aujla & Sensor accuracy testing\\
 & Microcontroller hardware \& code testing\\
 & Touchscreen UI latency tests\\
 \hline
 Jonathan Cels & Microcontroller code testing\\
 & Chess API integration testing\\
 & Communication protocol testing\\
 \hline
 Joshua Chapman& Power distribution design and integration testing\\
 & Microcontroller hardware \& code testing\\
 & Communication protocol testing\\
 \hline
 Rupinder Nagra & Chess API integration testing\\ 
 & Web application connection tests\\
 & Web application functionality testing\\
 \hline
\end{tabular}
%You, your classmates and the course instructor.  Maybe your supervisor.
%You shoud do more than list names.  You should say what each person's role is
%for the project.  A table is a good way to summarize this information.

\subsection{SRS Verification Plan}
SRS verification is performed by the teammates throughout the design and testing 
process. They will reference requirements and consider the results throughout design. 
Teammates will periodically verify this completeness throughout the design process. 
The verification and validation report will reference an associated requirement for 
each test. This allows for tracking of individual requirement fullfilment 
throughout the process. 
%List any approaches you intend to use for SRS verification.  This may just
%be ad hoc feedback from reviewers, like your classmates, or you may have
%something more rigorous/systematic in mind..

%Remember you have an SRS checklist

\subsection{Design Verification Plan}
\textbf{Hardware Design Verification} begins with LTSpice and Multisim design software
packages. The designed circuits are simulated in the software and tested accordingly.
Testingincludes simulating inputs and verifying that expected outputs are returned. 
Edge cases are simulated to verify safety and failure conditions of the circuits. 
\newline
\newline
\textbf{Software Design Verification} utilizes the compiler to verify code correctness. 
Arduino intergrated development environments contain the tools to compile and check for 
errors. Code walkthroughs are performed by collaborators that did not write the program. 
This includes detailed inspection and a report describing the function of the code 
based on their perspective. This tests the readability and functionality of 
the code. 
%Plans for design verification}
%The review will include reviews by your classmates}
%Remember you have MG and MIS checklists}

\subsection{Verification and Validation Plan Verification Plan}

The Verification and Validation Plan verification will include reviewing the \href{file:../Checklists/VnV-Checklist.pdf}{Verification and Validation Plan checklist} and adjusting this document accordingly. The verification will also include considering feedback from issues created by another team, as well as from the TA assigned to this project. Comments and improvements will be implemented after completing revision 0 of this document.
%The verification and validation plan is an artifact that should also be verified.

%The review will include reviews by your classmates

%Create a checklists?

\subsection{Implementation Verification Plan}
\textbf{Hardware Implementation Verification} begins with individual components. First 
the power supply is tested with a voltemeter for accuracy and precision. Then, 
the microcontroller is powered on and each individual I/O is tested for correctness. 
Inputs will be powered via the power supply and readings are verified for correctness 
using custom testing software installed on the controller. Outputs are verified using 
the voltemeter to measure correctness relative to the controller value. Finally, 
sensors and circuits are tested using the IO of the microcontroller. This allows
for accurate and detailed inspection of the components and verifies their correctness. 
\newline 
\newline 
\textbf{Software Implementation Verification} requires the hardware to be tested and 
assembled before beginning. The software is downloaded to the controller and the inputs 
and outputs are configured one-by-one. Functionality of code is tested with unit tests
from section 5.1 with the appropriate hardware. Once each section is unit tested, the
sub-systems can begin to combine to test completness of full systems and their functionality.

%You should at least point to the tests listed in this document and the unit
%testing plan.
%In this section you would also give any details of any plans for static verification of
%the implementation.  Potential techniques include code walkthroughs, code
%inspection, static analyzers, etc.

\subsection{Automated Testing and Verification Tools}

Automated testing will be carried out for the software involved with this project. Unit tests will be created to test the functionality fo each function created for the program. A minimum of one "successful" test will be written to describe the intended program execution, and one or more "unsuccessful" tests will be written to test the robustness of the program. This will ensure complete code coverage for the software. There are 3 main classes of tests that will be run involving the different aspects \progname{}:
\begin{itemize}
  \item \textbf{Linting:} Performed on Python and Javascript code. This will be integrated into VS Code to assist in local development; this will be run as a Github Workflow as a non-blocking check before building the software.

    \begin{itemize}
      \item \href{https://marketplace.visualstudio.com/items?itemName=dbaeumer.vscode-eslint}{ESLint} to be used for Javascript
      \item \href{https://marketplace.visualstudio.com/items?itemName=ms-python.flake8}{Flake8} to be used for Python
    \end{itemize}

  \item \textbf{Unit Tests:} Detailed in \hyperref[UnitTests]{Section \ref*{UnitTests}}, unit tests will be performed for Python, Javascript and C code.
    \begin{itemize}
      \item \href{https://github.com/testing-library/react-testing-library}{React Testing Library} will be used for Javascript unit tests.
      \item \href{https://docs.pytest.org/en/7.2.x/}{PyTest} will be used for Python unit tests.
      \item \href{https://aceunit.sourceforge.net/}{AceUnit} will be used for C unit tests.
    \end{itemize}

  \item \textbf{Dynamic Analysis Tool:} in addition to creating unit tests for the C code, memory leaks and access errors will be caught using \href{https://valgrind.org/}{ValGrind}.
  
\end{itemize}


\textbf{Debuggers:} Dynamic analysis will also be performed on the code through debuggers to verify the system and unit tests, as well as normal operation. These tools will be used extensively to bring all unit tests to a successful state, and to determine the root cause of any inconsistent behaviour from the hardware.
  \begin{itemize}
    \item The broswer's (Chrome, Firefox, etc.) built in debugger will be used for stepping through the Javascript code on the web application.
    \item \href{https://docs.python.org/3/library/pdb.html}{pdb} will be used for Python debugging.
    \item \href{https://docs.arduino.cc/tutorials/zero/debugging-with-zero}{Arduino Zero Built-in Debugger Interface} will be used for debugging embedded C code.
  \end{itemize}


%What tools are you using for automated testing.  Likely a unit testing framework and maybe a profiling tool, like ValGrind.  Other possible tools include a static analyzer, make, continuous integration tools, test coverage tools, etc.  Explain your plans for summarizing code coverage metrics. Linters are another important class of tools.  For the programming language you select, you should look at the available linters.  There may also be tools that verify that coding standards have been respected, like flake9 for Python.

%If you have already done this in the development plan, you can point to that document.

%The details of this section will likely evolve as you get closer to the implementation.



\subsection{Software Validation Plan}

Software will be validated by reviewing the requirements in the \href{file:../SRS/SRS.pdf}{Software Requirement Specifications} document and ensuring the software matches the expected performance and behaviour. The chess engine software integration will be validated by comparing it alongside an established online chess engine running the same Stockfish build as is selected for \progname{}, and comparing the recommended moves. One webside that may be used is \href{https://www.365chess.com/analysis_board.php}{365Chess.com}.
\newline
\newline
Feedback from members of the team that are assigned to another area of the system will be called to verify the software is performing adequately. For example, a member working on the web app may come to test the hardware to ensure the sequences and behaviour is intuitive and user friendly. Feedback and comments from the professor and TA assigned to this project will also be considered when making ajustments to the software.
\newline
\newline
Unit tests will be created that capture both normal-use and failure modes to validate the software as well. Unit tests will be integrated in a Github workflow to ensure that changes are never made to the software that conflict with other areas of the code. Failing unit tests will be used as direction to change parts of the software back to a valid state of operation.

%If there is any external data that can be used for validation, you should point to it here.  If there are no plans for validation, you should state that here.

%You might want to use review sessions with the stakeholder to check that the requirements document captures the right requirements.  Maybe task based inspection?

%This section might reference back to the SRS verification section.

\section{System Test Description} \label{SystemTests}
	
\subsection{Tests for Functional Requirements}

\wss{Subsets of the tests may be in related, so this section is divided into
  different areas.  If there are no identifiable subsets for the tests, this
  level of document structure can be removed.}

\wss{Include a blurb here to explain why the subsections below
  cover the requirements.  References to the SRS would be good here.}

\subsubsection{Area of Testing1}

\wss{It would be nice to have a blurb here to explain why the subsections below
  cover the requirements.  References to the SRS would be good here.  If a section
  covers tests for input constraints, you should reference the data constraints
  table in the SRS.}
		
\paragraph{Title for Test}

\begin{enumerate}

\item{test-id1\\}

Control: Manual versus Automatic
					
Initial State: 
					
Input: 
					
Output: \wss{The expected result for the given inputs}

Test Case Derivation: \wss{Justify the expected value given in the Output field}
					
How test will be performed: 
					
\item{test-id2\\}

Control: Manual versus Automatic
					
Initial State: 
					
Input: 
					
Output: \wss{The expected result for the given inputs}

Test Case Derivation: \wss{Justify the expected value given in the Output field}

How test will be performed: 

\end{enumerate}

\subsubsection{Area of Testing2}

...

\subsection{Tests for Nonfunctional Requirements}

\wss{The nonfunctional requirements for accuracy will likely just reference the
  appropriate functional tests from above.  The test cases should mention
  reporting the relative error for these tests.  Not all projects will
  necessarily have nonfunctional requirements related to accuracy}

\wss{Tests related to usability could include conducting a usability test and
  survey.  The survey will be in the Appendix.}

\wss{Static tests, review, inspections, and walkthroughs, will not follow the
format for the tests given below.}

\subsubsection{Look and Feel}
\paragraph{Style}
\begin{enumerate}
    \item{NFT1}

\item{test-id1\\}

Type: Functional, Dynamic, Manual, Static etc.
					
Initial State: 
					
Input/Condition: 
					
Output/Result: 
					
How test will be performed: 
					
\item{test-id2\\}

Type: Functional, Dynamic, Manual, Static etc.
					
Initial State: 
					
Input: 
					
Output: 
					
How test will be performed: 

\end{enumerate}

\subsubsection{Usability and Humanity}
\paragraph{Learnability}
\begin{enumerate}
    \item{NFT2}

        Type: Structural, Static, Manual
                            
        Initial State: Product is in normal mode, a game has started, and the users have not interacted with the product before. 
                            
        Input/Condition: Users are asked to use the product and move one specified piece from one square to another specified square on the board.
                            
        Output/Result: The majority of users understand which piece they moved and to where, and are able to identify that the web application 
            has reflected their move in the virtual model within 30 seconds of studying the visual representation of the board state.
                            
        How test will be performed: A test group of people who do not play chess regularly are asked to move pieces on the chessboard. They are 
            then asked to identify the move they just made as reflected in the web application model. The subjects must identify that the piece they
            moved on the board has also been moved on the web application's virtual model within 30 seconds or less, averaged over the number of people
            in the test group.
                        
    \item{NFT3}

        Type: Functional, Dynamic, Manual, Static etc.
                            
        Initial State: The product shall be representing the state of the game that is in progress. The pieces shall be in a legal position according to 
            the rules of chess. The pieces shall not all be in their starting positions, and there is at least one of each type of piece 
            (pawn, knight, bishop, rook, queen, king) on the board.
                            
        Input: Users are asked to identify the names of different pieces and squares based on their visual appearance in both the 
            physical product and the web application.
                            
        Output: The majority of users are able to identify the names of pieces and squares based on their likeliness to historically used symbols and shapes.
                            
        How test will be performed: A test group of people who have played chess in the past or play chess regularly are asked to identify each of the pieces 
            and squares from an in-progress game of chess using the system. The justification for this is to avoid piece identification based on their starting 
            positions. The majority of the group should be able to visually identify every piece and square within 2 minutes of seeing the position for the first time.
\end{enumerate}

\subsubsection{Performance}
\paragraph{Speed and Latency}
\begin{enumerate}
    \item{NFT4}

        Type: Structural, Static, Manual
                            
        Initial State: The product is in normal mode, and a game has started.
                            
        Input/Condition: Users are asked to pick up a piece and suspend it midair without placing it down.
                            
        Output/Result: The board shall visually indicate where the held piece is able to move according to the rules of chess within a specific time frame.
                            
        How test will be performed: A test group of people who have played chess in the past or play chess regularly are asked to pick up a specific piece and 
            hold it. The system will give visual indicators of where the held piece is able to move according to the rules of chess. The time between when 
            they pick up the piece and when the visual response occurs is measured and recorded. This process is repeated 5 times per user. The individual and average
            times are recorded. The response times are then averaged over the entire test group. The average response time of the entire test group must be less than 
            0.5 seconds.
                        
    \item{NFT5}

        Type: Structural, Static, Manual
                            
        Initial State: The results of the previous test, NFT4, have been measured and recorded.
                            
        Input/Condition: The results of NFT4.
                            
        Output/Result: The maximum recorded time of any individual response is within a specific time frame.
                            
        How test will be performed: The times measured in the previous test, NFT4, will be inspected. The maximum recorded individual time must be less than 
            1 second.

    \item{NFT6}

        Type: Structural, Static, Manual
                            
        Initial State: The product is in normal mode, and a game has started.
                            
        Input/Condition: Users are asked to pick up a piece and legally move it to a square according to the rules of chess.
                            
        Output/Result: The web application shall reflect their move in the virtual model within a specific time frame.
                            
        How test will be performed: A test group of people who have played chess in the past or play chess regularly are asked to pick up a specific piece and 
            legally move it to another square according to the rules of chess. The web application will reflect their move in the virtual model. The time between 
            when they place down the piece and when the web application response occurs is measured and recorded. This process is repeated 5 times per user. The 
            individual and average times are recorded. The response times are then averaged over the entire test group. The average response time of the entire
            test group must be less than 2 seconds.

    \item{NFT7}

        Type: Structural, Static, Manual
                                
        Initial State: The results of the previous test, NFT6, have been measured and recorded.
                            
        Input/Condition: The results of NFT6.
                            
        Output/Result: The maximum recorded time of any individual response is within a specific time frame.
                            
        How test will be performed: The times measured in the previous test, NFT6, will be inspected. The maximum recorded individual time must be less than 
            5 seconds.
\end{enumerate}

\paragraph{Health and Safety}
\begin{enumerate}
    \item{NFT8}

        Type: Structural, Static, Manual
                            
        Initial State: The product is in normal mode, and a game has started.
                            
        Input/Condition: 10 wires are chosen as a sample.
                            
        Output/Result: The maximum power on any single wire shall be within the required limit.
                            
        How test will be performed: A sample of 10 wires are chosen arbitrarily from across the entire system. The voltage and amperage of 
            each wire in the sample are measured and recorded. The power shall then be calculated and recorded. The maximum power of any wire
            in the sample must not exceed the safe limits determined in the Canadian Electrical Code \cite{CanadianElectricalCode2021}.
\end{enumerate}

\paragraph{Precision and Accuracy}
\begin{enumerate}
    \item{NFT9}

        Type: Structural, Static, Manual
                            
        Initial State: The product is in the initial game state.
                            
        Input/Condition: Users are instructed to play a full game of chess using the \progname{} system.
                            
        Output/Result: The web application will properly reflect the moves made on the physical product the majority of the time.
                            
        How test will be performed: A test group of people who have played chess in the past or play chess regularly are asked to play a full game of chess. 
            Their moves and the web application response will be recorded and compared against each other. The number of discrepancies between the physical 
            moves and moves made on the web application will be recorded. The number of discrepancies averaged over the entire test group must be less than 
            or equal to 1.
\end{enumerate}

\paragraph{Capacity}
\begin{enumerate}
    \item{NFT10}

        Type: Structural, Static, Manual
                            
        Initial State: The product is in the initial game state and is in engine mode.
                            
        Input/Condition: Moves are made until the game state is in one of a set of predetermined computationally complicated chess positions.
                            
        Output/Result: The level of memory used by the web application shall be no more than 1 Gigabyte (GB) at any measured point.
                            
        How test will be performed: Moves will be made until the game state is in one of a set of predetermined computationally complicated chess positions. 
            The engine will then be using the maximum amount of memory to compute the best possible moves for the position. The amount of memory used will be 
            measured and recorded using windows task manager. The amount of memory must never exceed 1 GB at any measured point.
\end{enumerate}

\subsubsection{Security}
\paragraph{Integrity}
\begin{enumerate}
    \item{NFT11}

        Type: Structural, Static, Manual
                            
        Initial State: The product is in normal mode and a chess game is in progress on the system.
                            
        Input/Condition: The Bluetooth connection is severed between the web application and the product.
                            
        Output/Result: The web application indicates that the Bluetooth connection has been lost.
                            
        How test will be performed: A game of chess is being played when the Bluetooth option is switched off on the server, severing the connection.
            The web application must display an alert that the Bluetooth connection has been lost.
\end{enumerate}

\begin{enumerate}
    \item{NFT12}

        Type: Structural, Static, Manual
                            
        Initial State: The product is in normal mode and a chess game is in progress on the system.
                            
        Input/Condition: The power connection to the system is severed.
                            
        Output/Result: The system stores the game state in local memory until power is restored.
                            
        How test will be performed: A game of chess is being played when the power is switched off. The power is then restored after 5 or more seconds. A single
            move is made and the state of the game is tested against the web application. The state of the game should be unchanged from before power was lost.
\end{enumerate}

\subsection{Traceability Between Test Cases and Requirements}

\wss{Provide a table that shows which test cases are supporting which
  requirements.}

\section{Unit Test Description} \label{UnitTests}

\wss{Reference your MIS (detailed design document) and explain your overall
  philosophy for test case selection.}  
\wss{This section should not be filled in until after the MIS (detailed design
  document) has been completed.}

\subsection{Unit Testing Scope}

\wss{What modules are outside of the scope.  If there are modules that are
  developed by someone else, then you would say here if you aren't planning on
  verifying them.  There may also be modules that are part of your software, but
  have a lower priority for verification than others.  If this is the case,
  explain your rationale for the ranking of module importance.}

\subsection{Tests for Functional Requirements}

\wss{Most of the verification will be through automated unit testing.  If
  appropriate specific modules can be verified by a non-testing based
  technique.  That can also be documented in this section.}

\subsubsection{Module 1}

\wss{Include a blurb here to explain why the subsections below cover the module.
  References to the MIS would be good.  You will want tests from a black box
  perspective and from a white box perspective.  Explain to the reader how the
  tests were selected.}

\begin{enumerate}

\item{test-id1\\}

Type: \wss{Functional, Dynamic, Manual, Automatic, Static etc. Most will
  be automatic}
					
Initial State: 
					
Input: 
					
Output: \wss{The expected result for the given inputs}

Test Case Derivation: \wss{Justify the expected value given in the Output field}

How test will be performed: 
					
\item{test-id2\\}

Type: \wss{Functional, Dynamic, Manual, Automatic, Static etc. Most will
  be automatic}
					
Initial State: 
					
Input: 
					
Output: \wss{The expected result for the given inputs}

Test Case Derivation: \wss{Justify the expected value given in the Output field}

How test will be performed: 

\item{...\\}
    
\end{enumerate}

\subsubsection{Module 2}

...

\subsection{Tests for Nonfunctional Requirements}

\wss{If there is a module that needs to be independently assessed for
  performance, those test cases can go here.  In some projects, planning for
  nonfunctional tests of units will not be that relevant.}

\wss{These tests may involve collecting performance data from previously
  mentioned functional tests.}

\subsubsection{Module ?}
		
\begin{enumerate}

\item{test-id1\\}

Type: \wss{Functional, Dynamic, Manual, Automatic, Static etc. Most will
  be automatic}
					
Initial State: 
					
Input/Condition: 
					
Output/Result: 
					
How test will be performed: 
					
\item{test-id2\\}

Type: Functional, Dynamic, Manual, Static etc.
					
Initial State: 
					
Input: 
					
Output: 
					
How test will be performed: 

\end{enumerate}

\subsubsection{Module ?}

...

\subsection{Traceability Between Test Cases and Modules}

\wss{Provide evidence that all of the modules have been considered.}
				
\bibliographystyle{plainnat}

\bibliography{../../refs/References}

\newpage

\section{Appendix}

This is where you can place additional information.

\subsection{Symbolic Parameters}

The definition of the test cases will call for SYMBOLIC\_CONSTANTS.
Their values are defined in this section for easy maintenance.

\subsection{Usability Survey Questions?}

\wss{This is a section that would be appropriate for some projects.}

\newpage{}
\section*{Appendix --- Reflection}

The information in this section will be used to evaluate the team members on the
graduate attribute of Lifelong Learning.  Please answer the following questions:

\newpage{}
\section*{Appendix --- Reflection}

The information in this section will be used to evaluate the team members on the
graduate attribute of Lifelong Learning.  Please answer the following questions:

\begin{enumerate}
  \item What knowledge and skills will the team collectively need to acquire to
  successfully complete the verification and validation of your project?
  Examples of possible knowledge and skills include dynamic testing knowledge,
  static testing knowledge, specific tool usage etc.  You should look to
  identify at least one item for each team member.
  \item For each of the knowledge areas and skills identified in the previous
  question, what are at least two approaches to acquiring the knowledge or
  mastering the skill?  Of the identified approaches, which will each team
  member pursue, and why did they make this choice?
\end{enumerate}

\end{document}