\documentclass[12pt, titlepage]{article}

\usepackage{booktabs}
\usepackage{tabularx}
\usepackage{hyperref}
\hypersetup{
    colorlinks,
    citecolor=blue,
    filecolor=black,
    linkcolor=red,
    urlcolor=blue
}
\usepackage[round]{natbib}

%% Comments

\usepackage{color}

\newif\ifcomments\commentstrue %displays comments
%\newif\ifcomments\commentsfalse %so that comments do not display

\ifcomments
\newcommand{\authornote}[3]{\textcolor{#1}{[#3 ---#2]}}
\newcommand{\todo}[1]{\textcolor{red}{[TODO: #1]}}
\else
\newcommand{\authornote}[3]{}
\newcommand{\todo}[1]{}
\fi

\newcommand{\wss}[1]{\authornote{blue}{SS}{#1}} 
\newcommand{\plt}[1]{\authornote{magenta}{TPLT}{#1}} %For explanation of the template
\newcommand{\an}[1]{\authornote{cyan}{Author}{#1}}

%% Common Parts

\newcommand{\progname}{Chess Connect} % PUT YOUR PROGRAM NAME HERE
\newcommand{\authname}{Team \#4,
\\ Alexander Van Kralingen
\\ Arshdeep Aujla
\\ Jonathan Cels
\\ Joshua Chapman
\\ Rupinder Nagra} % AUTHOR NAMES without MacIDs 

\usepackage{hyperref}
    \hypersetup{colorlinks=true, linkcolor=blue, citecolor=blue, filecolor=blue,
                urlcolor=blue, unicode=false}
    \urlstyle{same}
                                


\begin{document}

\title{Project Title: System Verification and Validation Plan for Chess Connect} 
\author{\authname}
\date{\today}
	
\maketitle

\pagenumbering{roman}

\section{Revision History}

\begin{tabularx}{\textwidth}{p{3cm}p{2cm}X}
\toprule {\bf Date} & {\bf Version} & {\bf Notes}\\
\midrule
October 31, 2022 & Arshdeep Aujla & Added section 3\\
Date 2 & 1.1 & Notes\\
\bottomrule
\end{tabularx}

\newpage

\tableofcontents

\listoftables
\wss{Remove this section if it isn't needed}

\listoffigures
\wss{Remove this section if it isn't needed}

\newpage

\section{Symbols, Abbreviations and Acronyms}

\renewcommand{\arraystretch}{1.2}
\begin{tabular}{l l} 
  \toprule		
  \textbf{symbol} & \textbf{description}\\
  \midrule 
  T & Test\\
  \bottomrule
\end{tabular}\\

\wss{symbols, abbreviations or acronyms -- you can simply reference the SRS
  \citep{SRS} tables, if appropriate}

\newpage

\pagenumbering{arabic}

This document ... \wss{provide an introductory blurb and roadmap of the
  Verification and Validation plan}

\section{General Information}

\subsection{Summary}
The project name is Chess Connect. It is comprised of software and hardware components. The hardware will consist of a reactive chess set 
connected to a microcontroller. The microcontroller will relay information on the chess board in the form of LEDs of the possible moves the user can make.
The software component of this project will consist of a web application that will reflect all of the chess piece's location on the physical board.

\subsection{Objectives}
The following objectives are the qualities that are the most important for the project.
\begin{itemize}
  \item The hardware should reflect relevent information on the LEDs on the chess board
  \item The software component should reflect the physical chess board in near-real time
  \item The movement of the chess pieces should be recorded by the hardware
\end{itemize}

\subsection{Relevant Documentation}
The following documents are relevent to this project.
\begin{itemize}
  \item SRS
  \item Hazard Analysis
  \item Requirements Document
  \item Design Document
  \item VnV Report
\end{itemize}

\section{Plan}

\wss{Introduce this section.   You can provide a roadmap of the sections to
  come.}

\subsection{Verification and Validation Team}

\wss{You, your classmates and the course instructor.  Maybe your supervisor.
  You shoud do more than list names.  You should say what each person's role is
  for the project.  A table is a good way to summarize this information.}

\subsection{SRS Verification Plan}

\wss{List any approaches you intend to use for SRS verification.  This may just
  be ad hoc feedback from reviewers, like your classmates, or you may have
  something more rigorous/systematic in mind..}

\wss{Remember you have an SRS checklist}

\subsection{Design Verification Plan}

\wss{Plans for design verification}

\wss{The review will include reviews by your classmates}

\wss{Remember you have MG and MIS checklists}

\subsection{Implementation Verification Plan}

\wss{You should at least point to the tests listed in this document and the unit
  testing plan.}

\wss{In this section you would also give any details of any plans for static verification of
  the implementation.  Potential techniques include code walkthroughs, code
  inspection, static analyzers, etc.}

\subsection{Automated Testing and Verification Tools}

\wss{What tools are you using for automated testing.  Likely a unit testing
  framework and maybe a profiling tool, like ValGrind.  Other possible tools
  include a static analyzer, make, continuous integration tools, test coverage
  tools, etc.  Explain your plans for summarizing code coverage metrics.
  Linters are another important class of tools.  For the programming language
  you select, you should look at the available linters.  There may also be tools
  that verify that coding standards have been respected, like flake9 for
  Python.}

\wss{The details of this section will likely evolve as you get closer to the
  implementation.}

\subsection{Software Validation Plan}

\wss{If there is any external data that can be used for validation, you should
  point to it here.  If there are no plans for validation, you should state that
  here.}

\section{System Test Description}
	
\subsection{Tests for Functional Requirements}

The following functional requirements are split into subsections 
that describe the Active and Inactive states of the
system, and the behaviour for the different user modes of the application.
Each of the user modes include distinct actions due to the purposes of each mode. 

  \subsubsection{Game Active State}
  
  \begin{enumerate}
  
  \item{GA-1\\}
  
  Type: Functional, Dynamic, Manual
                      
  Initial State: The game is in the Game Active State.
                      
  Input: The user will press the Resign/Draw button.
                      
  Output: The system will be changed to the Game Inactive State.
                      
  Test Case Derivation: The game shall be in the Game Inactive State due to the Resign/Draw button being pressed.
  
  How test will be performed: The function that changes the game state will be run using the appropriate inputs.
  After it has ran we will check to see if the game state has been modified.
  
  \item{GA-2\\}
  
  Type: Functional, Dynamic, Manual
                      
  Initial State: The game is in the Game Active State.
                      
  Input: The user will press the New Game button.
                      
  Output: The system will be unchanged.
                      
  Test Case Derivation: The game shall remain in the Game Active State after the New Game button is pressed.
  
  How test will be performed: The function that changes the game state will be run using the appropriate inputs.
  After it has ran we will check to see that the game state has not been modified.
  
  \item{GA-3\\}
  
  Type: Functional, Dynamic, Manual
                      
  Initial State: The game is in the Game Active State.
                      
  Input: The user will switch to one of the user modes (Normal Mode, Engine Mode, Beginner Mode).
  
  Output: The system will be changed to the selected user mode.
                      
  Test Case Derivation: The game shall be in the selected user mode due to the appropriate user mode switch being pressed.
  
  How test will be performed: The function that changes the user mode will be run using the appropriate inputs.
  After it has ran we will check to see if the user mode has been modified.
  
  % \item{GA-4\\}
  
  % Type: Functional, Dynamic, Manual
                      
  % Initial State: The game is in a user mode.
                      
  % Input: The game is changed to a different user mode.
  
  % Output: The system will follow the functional requirements 
  % section according to the currently selected user mode.
                      
  % Test Case Derivation: The game shall follow the instructions that are appropriate to the current user mode.
  
  % How test will be performed: The function that changes the user mode will be run using the appropriate inputs.
  % After it has ran we will check to see if the appropriate Chess Board section is being followed.
  
  % \item{GA-5\\}
  
  % Type: Functional, Dynamic, Manual
                      
  % Initial State: The game is in a user mode.
                      
  % Input: The game is changed to a different user mode.
  
  % Output: The system will follow the Data Transfer section based on the user mode.
                      
  % Test Case Derivation: The game shall follow the Data Transfer section based on the selected user mode.
  
  % How test will be performed: The function that changes the user mode will be run using the appropriate inputs.
  % After it has ran we will check to see if the appropriate Data Transfer section is being followed.
  
  \item{GA-6\\}
  
  Type: Functional, Dynamic, Manual
                      
  Initial State: The game is in the Game Inactive State.
                      
  Input: The game is changed to the Game Active State.
  
  Output: The game state will be reset to the starting position.
                      
  Test Case Derivation: The game shall be reset to the default starting position due to it entering the Game Active State.
  
  How test will be performed: The function that resets the starting position will be run using the appropriate inputs.
  After it has ran we will check to see if the board state has been reset to the starting position.
  
  \item{GA-7\\}
  
  Type: Functional, Dynamic, Manual
                      
  Initial State: The game is in the Game Active State.
                      
  Input: The game results in a stalemate or checkmate.
  
  Output: The game state will be changed to the Game Inactive State.
                      
  Test Case Derivation: The game shall be changed to the Game Inactive state due a stalemate or checkmate ending the game.
  
  How test will be performed: The function that modifies the game state will be run using the appropriate inputs.
  After it has ran we will check to see if the game state has been changed to the Game Inactive State based on the
  the checkmate or stalemate termination type.
  
  % \item{GA-8\\}
  
  % Type: Functional, Dynamic, Manual
                      
  % Initial State: The game is in a user mode.
                      
  % Input: The game is changed to a different user mode.
  
  % Output: The system will follow the Web Application section based on the user mode.
                      
  % Test Case Derivation: The game shall follow the Web Application section based on the selected user mode.
  
  % How test will be performed: The function that changes the user mode will be run using the appropriate inputs.
  % After it has ran we will check to see if the appropriate Web Application section is being followed.
  
  \end{enumerate}
  
  \subsubsection{Game Inactive State}
  
  \begin{enumerate}
  
    \item{GI-1\\}
  
    Type: Functional, Dynamic, Manual
                        
    Initial State: The game is in the Game Inactive State.
                        
    Input: The user will press the New Game button.
                        
    Output: The system will be changed to the Game Active State.
                        
    Test Case Derivation: The game shall be in the Game Active State due to the New Game button being pressed,
    starting the game at the default position.
    
    How test will be performed: The function that changes the game state will be run using the appropriate inputs.
    After it has ran we will check to see if the game state has been modified to the Game Active State.
  
    \item{GI-2\\}
  
    Type: Functional, Dynamic, Manual
                        
    Initial State: The game is in the Game Inactive State.
                        
    Input: The user will  try to switch to one of the user modes (Normal Mode, Engine Mode, Beginner Mode).
    
    Output: The system will be unchanged to the selected user mode.
                        
    Test Case Derivation: The game state shall be unchanged due to the user mode appropriate switch being pressed in the Game Inactive State.
    
    How test will be performed: The function that changes the user mode will be run using the appropriate inputs.
    After it has ran we will check to see if the game state has been unmodified.
  
    \item{GI-3\\}
  
    Type: Functional, Dynamic, Manual
                        
    Initial State: The game is in the Game Inactive State.
                        
    Input: The user will press the Resign/Draw button.
                        
    Output: The system will be unchanged.
                        
    Test Case Derivation: The game shall be in the Game Inactive State due to the Resign/Draw button having no effect.
    
    How test will be performed: The function that changes the game state will be run using the appropriate inputs.
    After it has ran we will check to see if the game state is unchanged.
  
    \item{GI-4\\}
  
    Type: Functional, Dynamic, Manual
                        
    Initial State: The game is in the Game Inactive State.
                        
    Input: The user will move a piece.
                        
    Output: The board state is not sent to the web application.
                        
    Test Case Derivation: The board state will not be sent due to the Resign/Draw button having no effect
    as it is in the Game Inactive State.
    
    How test will be performed: The function that changes the game state will be run using the appropriate inputs.
    After it has ran we will check to see if the game state is unchanged.
  
    \item{GI-5\\}
  
    Type: Functional, Dynamic, Manual
                        
    Initial State: The game is in the Game Active State.
                        
    Input: The game is terminated.
                        
    Output: The display the final game and message with the game termination type (stalemate,
    checkmate, resignation, draw).
                        
    Test Case Derivation: The game shall output the final game and message due to the game being terminated.
    
    How test will be performed: The function that handles actions after game termination will be run using the appropriate inputs.
    After it has ran we will check to see if the final game and message with termination type are displayed.
  
    \end{enumerate}
  
    \subsubsection{Normal Mode}
  
    \begin{enumerate}
  
      \item{NB-1\\}
  
      Type: Functional, Dynamic, Manual
                          
      Initial State: The game is in Normal Mode.
                          
      Input: A piece has been moved to a square.
                          
      Output: The system stores the position, colour, and type of piece in the micro-controller.
                          
      Test Case Derivation: The system should contain the position, colour, and type in the micro-controller
      after a piece has been moved to a square in order to track the board state for the web application.
  
      How test will be performed: The function that accesses the micro-controller will be run using the appropriate inputs.
      After it has ran we will check to see if the position, colour, and type of piece are stored in the micro-controller.  
  
      \item{NB-2\\}
  
      Type: Functional, Dynamic, Manual
                        
      Initial State: The game is in Normal Mode.
                          
      Input: The user will hold down the Resign button for ENDTIME seconds.
                          
      Output: The game state will change to the Game Inactive State.
                          
      Test Case Derivation: The game shall be in the Game Inactive State due to the Resign button being pressed.
      
      How test will be performed: The function that changes the game state will be run using the appropriate inputs.
      After it has ran we will check to see if the game state is in the Game Inactive State.
  
      \item{NB-3\\}
  
      Type: Functional, Dynamic, Manual
                        
      Initial State: The game is in Normal Mode.
                          
      Input: Both users will hold down the Draw button for ENDTIME seconds each located on their side of the board.
                          
      Output: The game state will change to the Game Inactive State.
                          
      Test Case Derivation: The game shall be in the Game Inactive State due to the Draw buttons being pressed.
      
      How test will be performed: The function that changes the game state will be run using the appropriate inputs.
      After it has ran we will check to see if the game state is in the Game Inactive State.
  
      \item{ND-1\\}
  
      Type: Functional, Dynamic, Manual
                        
      Initial State: The game is in Normal Mode.
                          
      Input: The system shall send the micro-controller output over the chosen data transfer
      method as an input to the web application.
                          
      Output: The web application shall receive the micro-controller output sent over the chosen data transfer
      method.
                          
      Test Case Derivation: The web application should receive new information regarding the board state over the chosen data transfer method. 
  
      How test will be performed: The function that checks to see if any information is being transmitted from the micro-controller
      will be run using the appropriate inputs. After it has ran we will check to see if the appropriate information has been received.
  
      \item{NA-2\\}
  
      Type: Functional, Dynamic, Manual
                        
      Initial State: The game is in Normal Mode.
                          
      Input: The web application shall receive the micro-controller output sent over the chosen data transfer
      method.
                          
      Output: The web application will display the updated game board configuration with the data
      of the previous move.
                          
      Test Case Derivation: The web application should update the board configuration after receiving new information about the board state. 
  
      How test will be performed: The function that updates the game board configuration
      will be run using the appropriate inputs. After it has ran we will check to see if the board state has been updated.
  
      \item{NA-3\\}
  
      Type: Functional, Dynamic, Manual
                        
      Initial State: The game is in Normal Mode.
                          
      Input: The game has been terminated through method of stalemate, checkmate, resignation, or draw.
                          
      Output: The web application will display a message with the method of game termination and the
      system shall change to the Game Inactive State.
                          
      Test Case Derivation: The web application should display a message signifying user of the method of game termination
      and the game should modify state if the game has ended.
  
      How test will be performed: The function that checks to see if any of the game termination method in effect
      will be run using the appropriate inputs. After it has ran we will output the termination method and change the
      game state to the Game Inactive State.
  
    \end{enumerate}
  
    \subsubsection{Engine Mode}
  
    \begin{enumerate}
  
      \item{EB-1\\}
  
      Type: Functional, Dynamic, Manual
                          
      Initial State: The game is in Engine Mode.
                          
      Input: A piece has been moved to a square.
                          
      Output: The system stores the position, colour, and type of piece in the micro-controller.
                          
      Test Case Derivation: The system should contain the position, colour, and type in the micro-controller
      after a piece has been moved to a square.
  
      How test will be performed: The function that accesses the micro-controller will be run using the appropriate inputs.
      After it has ran we will check to see if the position, colour, and type of piece are stored in the micro-controller.  
  
      \item{EB-2\\}
  
      Type: Functional, Dynamic, Manual
                        
      Initial State: The game is in Engine Mode.
                          
      Input: The user will hold down the Resign button for ENDTIME seconds.
                          
      Output: The game state will change to the Game Inactive State.
                          
      Test Case Derivation: The game shall be in the Game Inactive State due to the Resign button being pressed for ENDTIME seconds, signifying that the game should end.
      
      How test will be performed: The function that changes the game state will be run using the appropriate inputs.
      After it has ran we will check to see if the game state is in the Game Inactive State.
  
      \item{EB-3\\}
  
      Type: Functional, Dynamic, Manual
                        
      Initial State: The game is in Engine Mode.
                          
      Input: Both users will hold down the Draw button for ENDTIME seconds each located on their side of the board.
                          
      Output: The game state will change to the Game Inactive State.
                          
      Test Case Derivation: The game shall be in the Game Inactive State due to the Draw buttons being pressed.
      
      How test will be performed: The function that changes the game state will be run using the appropriate inputs.
      After it has ran we will check to see if the game state is in the Game Inactive State.
  
      \item{EB-4\\}
  
      Type: Functional, Dynamic, Manual
                        
      Initial State: The game is in Engine Mode.
                          
      Input: The top engine moves are transmitted to the LCD display from the web application.
                          
      Output: The LCD display shows the top engine moves.
                          
      Test Case Derivation: The top engine moves need to be transmitted
      to the LCD display in order to show both users the best moves for a position.
      
      How test will be performed: The function that displays characters on the LCD display will be run using the appropriate inputs.
      After it has ran we will check to see if the requested characters are correctly displayed.
  
      \item{ED-1\\}
  
      Type: Functional, Dynamic, Manual
                        
      Initial State: The game is in Engine Mode.
                          
      Input: The system shall send the micro-controller output over the chosen data transfer
      method as an input to the web application.
                          
      Output: The web application shall receive the micro-controller output sent over the chosen data transfer
      method.
                          
      Test Case Derivation: The web application should receive new information regarding the board state over the chosen data transfer method. 
  
      How test will be performed: The function that checks to see if any information is being transmitted from the micro-controller
      will be run using the appropriate inputs. After it has ran we will check to see if the appropriate information has been received.
  
      \item{ED-2\\}
  
      Type: Functional, Dynamic, Manual
                        
      Initial State: The game is in Engine Mode.
                          
      Input: The system shall send the web application engine moves to the LCD display over the
      chosen data transfer method.
                          
      Output: The LCD display shall receive the web application output sent over the chosen data transfer
      method.
                          
      Test Case Derivation: The LCD display should receive the top engine moves over the chosen data transfer method. 
  
      How test will be performed: The function that checks to see if any information is being transmitted from the web application 
      using the appropriate inputs. After it has ran we will check to see if the appropriate information has been received by the LCD display.
  
      \item{EA-2\\}
  
      Type: Functional, Dynamic, Manual
                        
      Initial State: The game is in Engine Mode.
                          
      Input: The web application shall receive the micro-controller output sent over the chosen data transfer
      method.
                          
      Output: The web application will display the updated game board configuration with the data
      of the previous move
                          
      Test Case Derivation: The web application should update the board configuration after receiving new information about the board state. 
  
      How test will be performed: The function that updates the game board configuration
      will be run using the appropriate inputs. After it has ran we will check to see if the board state has been updated.
  
      \item{EA-5\\}
  
      Type: Functional, Dynamic, Manual
                        
      Initial State: The game is in Engine Mode.
                          
      Input: The system shall use the chess engine to evaluate the position and calculate the best
      engine moves.
                          
      Output: The system shall display the calculated engine moves on the web application.
                          
      Test Case Derivation: The chess engine should update the board configuration after receiving new information about the board state. 
  
      How test will be performed: The function that calculates the engine moves
      will be run using the appropriate inputs. After it has ran we will check to see if the correct engine moves are played
      by comparing the results to the identical engine on an online platform.
  
      \item{EA-6\\}
  
      Type: Functional, Dynamic, Manual
                        
      Initial State: The game is in Engine Mode.
                          
      Input: The game has been terminated through method of stalemate, checkmate, resignation, or draw.
                          
      Output: The web application will display a message with the method of game termination and the
      system shall change to the Game Inactive State.
                          
      Test Case Derivation: The web application should display a message signifying user of the method of game termination
      and the game should modify state if the game has ended.
  
      How test will be performed: The function that checks to see if any of the game termination methods are in effect
      will be run using the appropriate inputs. After it has ran we will out the termination method of change the
      game state to the Game Inactive State.
  
    \end{enumerate}
     
    \subsubsection{Beginner Mode}
  
    \begin{enumerate}
  
      \item{BB-1\\}
  
      Type: Functional, Dynamic, Manual
                          
      Initial State: The game is in Beginner Mode.
                          
      Input: A piece has been moved to a square.
                          
      Output: The system stores the position, colour, and type of piece in the micro-controller.
                          
      Test Case Derivation: The system should contain the position, colour, and type in the micro-controller
      after a piece has been moved to a square.
  
      How test will be performed: The function that accesses the micro-controller will be run using the appropriate inputs.
      After it has ran we will check to see if the position, colour, and type of piece are stored in the micro-controller.  
  
      \item{BB-2\\}
  
      Type: Functional, Dynamic, Manual
                          
      Initial State: The game is in Beginner Mode.
                          
      Input: The user picks up a piece.
                          
      Output: Allow user to view all legal moves with green LED
      lights on valid squares.
                          
      Test Case Derivation: This is needed to ensure users provide legal inputs to the system,
      and also provide visual feedback to accelerate learning.
  
      How test will be performed: The function that accesses the legal moves and LEDs will be run using the appropriate inputs.
      After it has ran we will check to see if the correct legal moves are shown with green LED lights.  
  
      \item{BB-3\\}
  
      Type: Functional, Dynamic, Manual
                          
      Initial State: The game is in Beginner Mode.
                          
      Input: The player makes an illegal move.
                          
      Output: The tile that the piece is moved onto will display a red LED
      light.
                          
      Test Case Derivation: This is needed to prevent users from providing illegal inputs to the system
      to minimize errors, and also provide visual feedback to accelerate learning.
  
      How test will be performed: The function that accesses the legal moves and LEDs will be run using the appropriate inputs.
      After it has ran we will display a red LED light if the output is not within the legal moves.  
  
      \item{BB-4\\}
  
      Type: Functional, Dynamic, Manual
                        
      Initial State: The game is in Beginner Mode.
                          
      Input: The user will hold down the Resign button for ENDTIME seconds.
                          
      Output: The game state will change to the Game Inactive State.
                          
      Test Case Derivation: The game shall be in the Game Inactive State due to the Resign button being pressed.
      
      How test will be performed: The function that changes the game state will be run using the appropriate inputs.
      After it has ran we will check to see if the game state is in the Game Inactive State.
  
      \item{BB-5\\}
  
      Type: Functional, Dynamic, Manual
                        
      Initial State: The game is in Beginner Mode.
                          
      Input: Both users will hold down the Draw button for ENDTIME seconds each located on their side of the board.
                          
      Output: The game state will change to the Game Inactive State.
                          
      Test Case Derivation: The game shall be in the Game Inactive State due to the Draw buttons being pressed.
      
      How test will be performed: The function that changes the game state will be run using the appropriate inputs.
      After it has ran we will check to see if the game state is in the Game Inactive State.
  
      \item{BD-1\\}
  
      Type: Functional, Dynamic, Manual
                        
      Initial State: The game is in Beginner Mode.
                          
      Input: The system shall send the micro-controller output over the chosen data transfer
      method as an input to the web application.
                          
      Output: The web application shall receive the micro-controller output sent over the chosen data transfer
      method.
                          
      Test Case Derivation: The web application should receive new information regarding the board state over the chosen data transfer method. 
  
      How test will be performed: The function that checks to see if any information is being transmitted from the micro-controller
      will be run using the appropriate inputs. After it has ran we will check to see if the appropriate information has been received.
  
      \item{BA-1\\}
  
      Type: Functional, Dynamic, Manual
                        
      Initial State: The game is in Beginner Mode.
                          
      Input: The user clicks to view the instructions regarding the rules of chess.
                          
      Output: The web application displays a detailed set of rules on how to to play chess.
                          
      Test Case Derivation: TIt is necessary for users to have a set of instructions as reference while
      using the application in Beginner Mode.
  
      How test will be performed: The function that displays the rules
      will be run using the appropriate inputs. After it has ran we will check to see if the web application is displaying the rules.
  
      \item{BA-2\\}
  
      Type: Functional, Dynamic, Manual
                        
      Initial State: The game is in Beginner Mode.
                          
      Input: The web application shall receive the micro-controller output sent over the chosen data transfer
      method.
                          
      Output: The web application will display the updated game board configuration with the data
      of the previous move
                          
      Test Case Derivation: The web application should update the board configuration after receiving new information about the board state. 
  
      How test will be performed: The function that updates the game board configuration
      will be run using the appropriate inputs. After it has ran we will check to see if the board state has been updated.
  
    \end{enumerate}

\subsection{Tests for Nonfunctional Requirements}

\wss{The nonfunctional requirements for accuracy will likely just reference the
  appropriate functional tests from above.  The test cases should mention
  reporting the relative error for these tests.}

\wss{Tests related to usability could include conducting a usability test and
  survey.}

\subsubsection{Area of Testing1}
		
\paragraph{Title for Test}

\begin{enumerate}

\item{test-id1\\}

Type: 
					
Initial State: 
					
Input/Condition: 
					
Output/Result: 
					
How test will be performed: 
					
\item{test-id2\\}

Type: Functional, Dynamic, Manual, Static etc.
					
Initial State: 
					
Input: 
					
Output: 
					
How test will be performed: 

\end{enumerate}

\subsubsection{Area of Testing2}

...

\subsection{Traceability Between Test Cases and Requirements}

\wss{Provide a table that shows which test cases are supporting which
  requirements.}

\section{Unit Test Description}

\wss{Reference your MIS and explain your overall philosophy for test case
  selection.}  
\wss{This section should not be filled in until after the MIS has
  been completed.}

\subsection{Unit Testing Scope}

\wss{What modules are outside of the scope.  If there are modules that are
  developed by someone else, then you would say here if you aren't planning on
  verifying them.  There may also be modules that are part of your software, but
  have a lower priority for verification than others.  If this is the case,
  explain your rationale for the ranking of module importance.}

\subsection{Tests for Functional Requirements}




\subsection{Tests for Nonfunctional Requirements}

\wss{If there is a module that needs to be independently assessed for
  performance, those test cases can go here.  In some projects, planning for
  nonfunctional tests of units will not be that relevant.}

\wss{These tests may involve collecting performance data from previously
  mentioned functional tests.}

\subsubsection{Module ?}
		
\begin{enumerate}

\item{test-id1\\}

Type: \wss{Functional, Dynamic, Manual, Automatic, Static etc. Most will
  be automatic}
					
Initial State: 
					
Input/Condition: 
					
Output/Result: 
					
How test will be performed: 
					
\item{test-id2\\}

Type: Functional, Dynamic, Manual, Static etc.
					
Initial State: 
					
Input: 
					
Output: 
					
How test will be performed: 

\end{enumerate}

\subsubsection{Module ?}

...

\subsection{Traceability Between Test Cases and Modules}

\wss{Provide evidence that all of the modules have been considered.}
				
\bibliographystyle{plainnat}

\bibliography{../../refs/References}

\newpage

\section{Appendix}

This is where you can place additional information.

\subsection{Symbolic Parameters}

The definition of the test cases will call for SYMBOLIC\_CONSTANTS.
Their values are defined in this section for easy maintenance.

\subsection{Usability Survey Questions?}

\wss{This is a section that would be appropriate for some projects.}

\end{document}