\documentclass[12pt]{article}
\usepackage[utf8]{inputenc}
\usepackage{tabularx}
\usepackage{booktabs}
\usepackage{siunitx}

%% Comments

\usepackage{color}

\newif\ifcomments\commentstrue %displays comments
%\newif\ifcomments\commentsfalse %so that comments do not display

\ifcomments
\newcommand{\authornote}[3]{\textcolor{#1}{[#3 ---#2]}}
\newcommand{\todo}[1]{\textcolor{red}{[TODO: #1]}}
\else
\newcommand{\authornote}[3]{}
\newcommand{\todo}[1]{}
\fi

\newcommand{\wss}[1]{\authornote{blue}{SS}{#1}} 
\newcommand{\plt}[1]{\authornote{magenta}{TPLT}{#1}} %For explanation of the template
\newcommand{\an}[1]{\authornote{cyan}{Author}{#1}}

%% Common Parts

\newcommand{\progname}{Chess Connect} % PUT YOUR PROGRAM NAME HERE
\newcommand{\authname}{Team \#4,
\\ Alexander Van Kralingen
\\ Arshdeep Aujla
\\ Jonathan Cels
\\ Joshua Chapman
\\ Rupinder Nagra} % AUTHOR NAMES without MacIDs 

\usepackage{hyperref}
    \hypersetup{colorlinks=true, linkcolor=blue, citecolor=blue, filecolor=blue,
                urlcolor=blue, unicode=false}
    \urlstyle{same}
                                


\begin{document}

\title{Software Requirements Specification for Chess Connect: Online tools combined with on-board vision to improve and share your game} 
\author{\authname}
\date{October 4th, 2022}
	
\maketitle

~\newpage

\tableofcontents

~\newpage

\addcontentsline{toc}{section}{Table of Revisions}
\section*{Table of Revisions}
\begin{table}[hp]
\caption{Revision History} \label{TblRevisionHistory}
\begin{tabularx}{\textwidth}{llX}
\toprule
\textbf{Date} & \textbf{Developer(s)} & \textbf{Change}\\
\midrule
2022-10-04 & Jonathan Cels & Template creation and document formatting\\ 
date & name & change\\
\bottomrule
\end{tabularx}
\end{table}

~\newpage

\section{Units, Terms, Acronyms, and Abbreviations}

\subsection{Table of Units}
Throughout this document SI (Syst\`{e}me International d'Unit\'{e}s) is employed
as the unit system.  In addition to the basic units, several derived units are
used as described below.  For each unit, the symbol is given followed by a
description of the unit and the SI name.

\begin{table}[ht]
  \noindent \begin{tabular}{l l l} 
    \toprule		
    \textbf{symbol} & \textbf{unit} & \textbf{SI}\\
    \midrule 
    \si{\volt} & electric potential & volt\\
    \si{\ampere} & current	& ampere\\
    \si{\ohm} & resistance	& ohm\\
    \si{\second} & time & second\\
    \si{\celsius} & temperature & centigrade\\
    \si{\joule} & energy & joule\\
    \si{\watt} & power & watt (W = \si{\joule\per\second})\\
    \bottomrule
  \end{tabular}
\end{table}

\newpage

\subsection{Abbreviations and Acronyms}
\begin{tabular}{l l} 
  \toprule		
  \textbf{symbol} & \textbf{description}\\
  \midrule 
  A & Assumption\\
  DD & Data Definition\\
  GD & General Definition\\
  GS & Goal Statement\\
  IM & Instance Model\\
  LC & Likely Change\\
  LCD & Liquid Crystal Display\\
  LED & Light-Emmitting Diode\\
  MCU & Micro Controller Unit\\
  PS & Physical System Description\\
  R & Requirement\\
  SRS & Software Requirements Specification\\
  T & Theoretical Model\\
  \bottomrule
\end{tabular}\\

\subsection{Mathematical Notation}

\subsection{Terminology and  Definitions}

\section{Introduction}
\subsection{Document Purpose}
\subsection{Characteristics of Intended Reader}
\subsection{Characteristics of Intended User}
\subsection{Stakeholders}

\section{Problem Description}

\section{Assumptions}

\section{Constraints}

\section{Scope}

\section{Project Overview}
\subsection{System Context Diagram}
\subsection{Normal Operation}
\subsubsection{Description}
\subsubsection{Use Cases/Scenarios}

\subsection{Behaviour Overview}
\subsection{Undesired Scenario Handling}

\section{System Level Variables}
\subsection{Constants}
\subsection{Monitored Variables}
\subsection{Controlled Variables}

~\newpage

\section{Requirements}
\subsection{Functional Requirements}
\subsection{Nonfunctional Requirements}

\newcounter{vnvSectionNfr}
\setcounter{vnvSectionNfr}{1}

\newcounter{nfrNum}
\setcounter{nfrNum}{1}

\subsubsection{Look and Feel Requirements}
\noindent \begin{enumerate}
    \item Example
\end{enumerate}
\subsubsection{Usability and Humanity Requirements}
\subsubsection{Performance Requirements}
\subsubsection{Operational and Environmental Requirements}
\subsubsection{Maintainability and Support Requirements}
\subsubsection{Security Requirements}
\subsubsection{Cultural and Political Requirements}
\subsubsection{Legal Requirements}

\noindent \begin{itemize}

% \item[NFR\thenfrNum \stepcounter{nfrNum} \label{NFR_Accuracy}:] \textbf{Accuracy}
%     {The software application game state will model the game state on the \progname{} hardware with a high degree of accuracy. 
%     The level of accuracy shall be described following the procedure given in Section 5.2.\thevnvSectionNfr~of the VnV (Verification and Validation) Plan.}
%     \stepcounter{vnvSectionNfr}
 
% \item[NFR\thenfrNum \stepcounter{nfrNum} \label{NFR_Usability}:] \textbf{Usability}
%     {The product will be able to be used by chess players of any experience level with minimal instruction.
%     The level of usability achieved by the system shall be described following the procedure given in Section 5.2.\thevnvSectionNfr~of the VnV Plan.}
%     \stepcounter{vnvSectionNfr}

% \item[NFR\thenfrNum \stepcounter{nfrNum} \label{NFR_Style}:] \textbf{Style}
%     {The product shall look and feel similar enough to traditional chess boards and chess pieces that the target audience will recognize the product 
%     as a chess set when encountering it for the first time. The level and speed of audience recognition achieved by the design shall be described 
%     following the procedure given in Section 5.2.\thevnvSectionNfr~of the VnV Plan.}
%     \stepcounter{vnvSectionNfr}

% \item[NFR\thenfrNum \stepcounter{nfrNum} \label{NFR_Maintainability}:] \textbf{Maintainability} 
%     {The effort required to make any of the likely changes listed for \progname{} should be less than FRACTION of the original development time.}

% \item[NFR\thenfrNum \stepcounter{nfrNum} \label{NFR_Portability}:] \textbf{Portability} 
%     {\progname{} shall be accessible from any web browser. The application shall be able to be hosted on at least any of the following systems:
%     \begin{enumerate}
%         \item Windows 8
%         \item Windows 10
%         \item Windows 11
%         \item WSL 2 (Windows Subsystem for Linux)
%         \item Ubuntu
%     \end{enumerate}}

% \item[NFR\thenfrNum \stepcounter{nfrNum} \label{NFR_Reusability}:] \textbf{Reusability}
 \item Other NFRs that might be discussed include verifiability,
%   understandability and reusability.
\end{itemize}

\section{Likely Changes}
\section{Unlikely Changes}

\section{Traceability Matrix}

\appendix
\section{Values of Auxiliary Constants}

\newpage

\appendix
\section{Reflection}
\subsection{Skills for Success}
\subsection{Knowledge and Learning Approaches}
\end{document}