\documentclass[12pt]{article}
\usepackage[utf8]{inputenc}
\usepackage{tabularx}
\usepackage{booktabs}
\usepackage{siunitx}
\usepackage[round]{natbib}

%% Comments

\usepackage{color}

\newif\ifcomments\commentstrue %displays comments
%\newif\ifcomments\commentsfalse %so that comments do not display

\ifcomments
\newcommand{\authornote}[3]{\textcolor{#1}{[#3 ---#2]}}
\newcommand{\todo}[1]{\textcolor{red}{[TODO: #1]}}
\else
\newcommand{\authornote}[3]{}
\newcommand{\todo}[1]{}
\fi

\newcommand{\wss}[1]{\authornote{blue}{SS}{#1}} 
\newcommand{\plt}[1]{\authornote{magenta}{TPLT}{#1}} %For explanation of the template
\newcommand{\an}[1]{\authornote{cyan}{Author}{#1}}

%% Common Parts

\newcommand{\progname}{Chess Connect} % PUT YOUR PROGRAM NAME HERE
\newcommand{\authname}{Team \#4,
\\ Alexander Van Kralingen
\\ Arshdeep Aujla
\\ Jonathan Cels
\\ Joshua Chapman
\\ Rupinder Nagra} % AUTHOR NAMES without MacIDs 

\usepackage{hyperref}
    \hypersetup{colorlinks=true, linkcolor=blue, citecolor=blue, filecolor=blue,
                urlcolor=blue, unicode=false}
    \urlstyle{same}
                                


\begin{document}

\title{Software Requirements Specification for \progname{}: Online tools combined with on-board vision to improve and share your game} 
\author{\authname}
\date{October 4th, 2022}
	
\maketitle

~\newpage

\tableofcontents

~\newpage

\addcontentsline{toc}{section}{Table of Revisions}
\section*{Table of Revisions}
\begin{table}[hp]
\caption{Revision History} \label{TblRevisionHistory}
\begin{tabularx}{\textwidth}{llX}
\toprule
\textbf{Date} & \textbf{Developer(s)} & \textbf{Change}\\
\midrule
2022-10-04 & Jonathan Cels & Template creation and document formatting\\ 
date & name & change\\
\bottomrule
\end{tabularx}
\end{table}

~\newpage

\section{Units, Terms, Acronyms, and Abbreviations}

\subsection{Table of Units}
Throughout this document SI (Syst\`{e}me International d'Unit\'{e}s) is employed
as the unit system.  In addition to the basic units, several derived units are
used as described below.  For each unit, the symbol is given followed by a
description of the unit and the SI name.

\begin{table}[ht]
  \noindent \begin{tabular}{l l l} 
    \toprule		
    \textbf{symbol} & \textbf{unit} & \textbf{SI}\\
    \midrule 
    \si{\volt} & electric potential & volt\\
    \si{\ampere} & current	& ampere\\
    \si{\ohm} & resistance	& ohm\\
    \si{\second} & time & second\\
    \si{\celsius} & temperature & centigrade\\
    \si{\joule} & energy & joule\\
    \si{\watt} & power & watt (W = \si{\joule\per\second})\\
    \bottomrule
  \end{tabular}
\end{table}

\newpage

\subsection{Abbreviations and Acronyms}
\begin{tabular}{l l} 
  \toprule		
  \textbf{symbol} & \textbf{description}\\
  \midrule 
  A & Assumption\\
  DD & Data Definition\\
  GD & General Definition\\
  GS & Goal Statement\\
  IM & Instance Model\\
  LC & Likely Change\\
  LCD & Liquid Crystal Display\\
  LED & Light-Emmitting Diode\\
  MCU & Micro Controller Unit\\
  PS & Physical System Description\\
  R & Requirement\\
  SRS & Software Requirements Specification\\
  T & Theoretical Model\\
  \bottomrule
\end{tabular}\\

\subsection{Mathematical Notation}

\subsection{Terminology and  Definitions}

\section{Introduction}
\subsection{Document Purpose}
\subsection{Characteristics of Intended Reader}
{The document is written with the purpose of guiding development for the \progname{} team. The intended readers of this document 
are the developers of \progname{}, Dr.~Spencer Smith, and Nicholas Annable, the teaching assistant assigned to this project. The 
document is thus written for an audience that is well-versed in formal specification at a university level. This includes models, 
diagrams, and mathematical notation. Readers should also have a university-level understanding of electrical circuit knowledge.}

\subsection{Characteristics of Intended User}
\subsection{Stakeholders}
{This project will assist chess players of any level that are looking for a tool to help them learn and study the game. For beginners, 
the board serves as a learning tutorial and a general introduction to the game, while intermediate and advanced players can use the 
engine move recommendations to study new lines, puzzles, and specific positions. In addition to chess enthusiasts, this project will 
also be relevant to chess tournament organizers looking for a method to easily broadcast and share their games online in real-time. }

\section{Problem Description}

\section{Assumptions}

\section{Constraints}

\section{Scope}
{The system is called \progname{}, and will include a software application and physical hardware device. The hardware will take the 
form of a chess set, and will collect and relay move and piece data. The device will convey the best moves for the specific board 
position, and will convey legal moves for specific pieces.The device will be connected to the software application, relaying and 
receiving relevant data. The software application will model and track the physical device, and will broadcast the data in an accessible 
format. The application will be constrained to a 2-dimensional model of the hardware device, showing a top-down view of the game.}

\bigskip
\noindent{\textbf{In-scope} items for the system include the following:
\begin{enumerate}
    \item Modeling and tracking a chess game played using the \progname{} hardware
    \item Displaying and broadcasting the game state on the \progname{} software application
    \item Giving users an option to choose between beginner mode, engine mode, and normal mode
    \begin{itemize}
        \item Beginner mode will display legal moves for individual pieces when a chess piece is picked up, and will warn the players when an illegal move is made
        \item Engine mode will display the best moves as determined by a chess engine for the position
        \item Normal mode will disable the engine and beginner mode features. This is intended for regular play between experienced players
    \end{itemize}
\end{enumerate}}

\bigskip

\noindent{The following items are deemed to be \textbf{out of scope}:
\begin{enumerate}
    \item FIDE (International Chess Federation) standards for tournament appropriate chess equipment
    \item Tracking and support for alternate chess variants such as Chess960, Atomic Chess, King of the Hill. More information found here: \cite{ListOfChessVariants2022}.
    \item Proper tracking of alternate starting positions like puzzles
    \item Proper tracking of illegal moves and rule violations when warnings are ignored
\end{enumerate}}

\section{Project Overview}
\subsection{System Context Diagram}
\subsection{Normal Operation}
\subsubsection{Description}
\subsubsection{Use Cases/Scenarios}

\subsection{Behaviour Overview}
\subsection{Undesired Scenario Handling}

\section{System Level Variables}
\subsection{Constants}
\subsection{Monitored Variables}
\subsection{Controlled Variables}

\section{Requirements}
\subsection{Functional Requirements}
\subsection{Nonfunctional Requirements}

\section{Likely Changes}
\section{Unlikely Changes}

\section{Traceability Matrix}

\appendix
\section{Values of Auxiliary Constants}

\newpage

\appendix
\section{Reflection}
\subsection{Skills for Success}
\subsection{Knowledge and Learning Approaches}

\bibliographystyle {plainnat}
\bibliography {../../refs/References}
\end{document}