\documentclass[12pt]{article}
\usepackage[utf8]{inputenc}
\usepackage{tabularx}
\usepackage{booktabs}
\usepackage{siunitx}

\title{SRS}
\author{celsj }
\date{October 2022}

\begin{document}

\maketitle

~\newpage

\tableofcontents

~\newpage

\addcontentsline{toc}{section}{Table of Revisions}
\section*{Table of Revisions}
\begin{table}[hp]
\caption{Revision History} \label{TblRevisionHistory}
\begin{tabularx}{\textwidth}{llX}
\toprule
\textbf{Date} & \textbf{Developer(s)} & \textbf{Change}\\
\midrule
2022-10-04 & Jonathan Cels & Template creation and document formatting\\ 
date & name & change\\
\bottomrule
\end{tabularx}
\end{table}

~\newpage

\section{Units, Terms, Acronyms, and \\Abbreviations}

\subsection{Table of Units}
Throughout this document SI (Syst\`{e}me International d'Unit\'{e}s) is employed
as the unit system.  In addition to the basic units, several derived units are
used as described below.  For each unit, the symbol is given followed by a
description of the unit and the SI name.
~\newline

\renewcommand{\arraystretch}{1.2}
\begin{table}[ht]
  \noindent \begin{tabular}{l l l} 
    \toprule		
    \textbf{symbol} & \textbf{unit} & \textbf{SI}\\
    \midrule 
    \si{\volt} & electric potential & volt\\
    \si{\ampere} & current	& ampere\\
    \si{\ohm} & resistance	& ohm\\
    \si{\second} & time & second\\
    \si{\celsius} & temperature & centigrade\\
    \si{\joule} & energy & joule\\
    \si{\watt} & power & watt (W = \si{\joule\per\second})\\
    \bottomrule
  \end{tabular}
\end{table}

\subsection{Abbreviations and Acronyms}
\renewcommand{\arraystretch}{1.2}
\begin{tabular}{l l} 
  \toprule		
  \textbf{symbol} & \textbf{description}\\
  \midrule 
  A & Assumption\\
  DD & Data Definition\\
  GD & General Definition\\
  GS & Goal Statement\\
  IM & Instance Model\\
  LC & Likely Change\\
  LCD & Liquid Crystal Display\\
  LED & Light-Emmitting Diode\\
  MCU & Micro Controller Unit\\
  PS & Physical System Description\\
  R & Requirement\\
  SRS & Software Requirements Specification\\
  T & Theoretical Model\\
  \bottomrule
\end{tabular}\\

\subsection{Mathematical Notation}

\subsubsection{Terminology and  Definitions}

\section{Introduction}
\subsection{Document Purpose}
\subsection{Characteristics of Intended Reader}
\subsection{Characteristics of Intended User}
\subsection{Stakeholders}

\section{Problem Description}


\section{Assumptions}

\section{Constraints}

\section{Scope}

\section{Project Overview}
\subsection{System Context Diagram}
\subsection{Normal Operation}
\subsubsection{Description}
\subsubsection{Use Cases/Scenarios}

\subsection{Behaviour Overview}
\subsection{Undesired Scenario Handling}

\section{System Level Variables}
\subsection{Constants}
\subsection{Monitored Variables}
\subsection{Controlled Variables}

\section{Requirements}
\subsection{Functional Requirements}
\subsection{Nonfunctional Requirements}

\section{Likely Changes}
\section{Unlikely Changes}

\section{Traceability Matrix}

\appendix
\section{Values of Auxiliary Constants}

\newpage

\appendix
\section{Reflection}
\subsection{Skills for Success}
\subsection{Knowledge and Learning Approaches}
\end{document}