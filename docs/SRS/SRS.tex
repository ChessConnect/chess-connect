\documentclass[12pt]{article}

\usepackage{amsmath, mathtools}
\usepackage{amsfonts}
\usepackage{amssymb}
\usepackage{graphicx}
\usepackage{colortbl}
\usepackage{xr}
\usepackage{hyperref}
\usepackage{longtable}
\usepackage{xfrac}
\usepackage{tabularx}
\usepackage{float}
\usepackage{siunitx}
\usepackage{booktabs}
\usepackage{caption}
\usepackage{pdflscape}
\usepackage{afterpage}
\usepackage{fullpage}
\usepackage{titlesec}
\usepackage[shortlabels]{enumitem}
\usepackage[round]{natbib}
\usepackage{array}

%% Comments

\usepackage{color}

\newif\ifcomments\commentstrue %displays comments
%\newif\ifcomments\commentsfalse %so that comments do not display

\ifcomments
\newcommand{\authornote}[3]{\textcolor{#1}{[#3 ---#2]}}
\newcommand{\todo}[1]{\textcolor{red}{[TODO: #1]}}
\else
\newcommand{\authornote}[3]{}
\newcommand{\todo}[1]{}
\fi

\newcommand{\wss}[1]{\authornote{blue}{SS}{#1}} 
\newcommand{\plt}[1]{\authornote{magenta}{TPLT}{#1}} %For explanation of the template
\newcommand{\an}[1]{\authornote{cyan}{Author}{#1}}

%% Common Parts

\newcommand{\progname}{Chess Connect} % PUT YOUR PROGRAM NAME HERE
\newcommand{\authname}{Team \#4,
\\ Alexander Van Kralingen
\\ Arshdeep Aujla
\\ Jonathan Cels
\\ Joshua Chapman
\\ Rupinder Nagra} % AUTHOR NAMES without MacIDs 

\usepackage{hyperref}
    \hypersetup{colorlinks=true, linkcolor=blue, citecolor=blue, filecolor=blue,
                urlcolor=blue, unicode=false}
    \urlstyle{same}
                                


\setcounter{secnumdepth}{4}

\titleformat{\paragraph}
{\normalfont\normalsize\bfseries}{\theparagraph}{1em}{}
\titlespacing*{\paragraph}
{0pt}{3.25ex plus 1ex minus .2ex}{1.5ex plus .2ex}

\begin{document}

\title{Software Requirements Specification for \progname{}: Online tools combined with on-board vision to improve and share your game} 
\author{\authname}
\date{October 4th, 2022}
	
\maketitle

~\newpage

\tableofcontents

~\newpage

\addcontentsline{toc}{section}{Table of Revisions}
\section*{Table of Revisions}
\begin{table}[hp]
\caption{Revision History} \label{TblRevisionHistory}
\begin{tabularx}{\textwidth}{llX}
\toprule
\textbf{Date} & \textbf{Developer(s)} & \textbf{Change}\\
\midrule
2022-10-04 & Jonathan Cels & Template creation and document formatting\\ 
2022-10-04 & Jonathan Cels & Non-functional requirements\\
2022-10-05 & Jonathan Cels & Scope, Intended Reader, Stakeholders\\
date & name & change\\
\bottomrule
\end{tabularx}
\end{table}

~\newpage

\section{Units, Terms, Acronyms, and Abbreviations}

\subsection{Table of Units}
Throughout this document SI (Syst\`{e}me International d'Unit\'{e}s) is employed
as the unit system.  In addition to the basic units, several derived units are
used as described below.  For each unit, the symbol is given followed by a
description of the unit and the SI name.

\begin{table}[ht]
  \noindent \begin{tabular}{l l l} 
    \toprule		
    \textbf{symbol} & \textbf{unit} & \textbf{SI}\\
    \midrule 
    \si{\volt} & electric potential & volt\\
    \si{\ampere} & current	& ampere\\
    \si{\ohm} & resistance	& ohm\\
    \si{\second} & time & second\\
    \si{\celsius} & temperature & centigrade\\
    \si{\joule} & energy & joule\\
    \si{\watt} & power & watt (W = \si{\joule\per\second})\\
    \bottomrule
  \end{tabular}
\end{table}

\newpage

\subsection{Abbreviations and Acronyms}
\begin{tabular}{l l} 
  \toprule		
  \textbf{symbol} & \textbf{description}\\
  \midrule 
  A & Assumption\\
  CSA & Canadian Standards Association\\
  DD & Data Definition\\
  FIDE & International Chess Federation or Fédération Internationale des Échecs\\
  GD & General Definition\\
  GS & Goal Statement\\
  IM & Instance Model\\
  LC & Likely Change\\
  LCD & Liquid Crystal Display\\
  LED & Light-Emmitting Diode\\
  MCU & Micro Controller Unit\\
  PS & Physical System Description\\
  R & Requirement\\
  SRS & Software Requirements Specification\\
  T & Theoretical Model\\
  VnV & Verification and Validation\\
  WCAG & Web Content Accessibility Guidelines\\
  \bottomrule
\end{tabular}\\

\subsection{Mathematical Notation}

\subsection{Terminology and  Definitions}

\section{Introduction}
\subsection{Document Purpose}
\subsection{Characteristics of Intended Reader}
{The document is written with the purpose of guiding development for the \progname{} team. The intended readers of this document 
are the developers of \progname{}, Dr.~Spencer Smith, and Nicholas Annable, the teaching assistant assigned to this project. The 
document is thus written for an audience that is well-versed in formal specification at a university level. This includes models, 
diagrams, and mathematical notation. Readers should also have a university-level understanding of electrical circuit knowledge.}

\subsection{Characteristics of Intended User}
\subsection{Stakeholders}
{This project will assist chess players of any level that are looking for a tool to help them learn and study the game. For beginners, 
the board serves as a learning tutorial and a general introduction to the game, while intermediate and advanced players can use the 
engine move recommendations to study new lines, puzzles, and specific positions. In addition to chess enthusiasts, this project will 
also be relevant to chess tournament organizers looking for a method to easily broadcast and share their games online in real-time. }

\section{Problem Description}

\section{Assumptions}

\section{Constraints}

\section{Scope}
{The system is called \progname{}, and will include a software application and physical hardware device. The hardware will take the 
form of a chess set, and will collect and relay move and piece data. The device will convey the best moves for the specific board 
position, and will convey legal moves for specific pieces.The device will be connected to the software application, relaying and 
receiving relevant data. The software application will model and track the physical device, and will broadcast the data in an accessible 
format. The application will be constrained to a 2-dimensional model of the hardware device, showing a top-down view of the game.}

\bigskip
\noindent{\textbf{In-scope} items for the system include the following:
\begin{enumerate}
    \item Modeling and tracking a chess game played using the \progname{} hardware
    \item Displaying and broadcasting the game state on the \progname{} software application
    \item Giving users an option to choose between beginner mode, engine mode, and normal mode
    \begin{itemize}
        \item Beginner mode will display legal moves for individual pieces when a chess piece is picked up, and will warn the players when an illegal move is made
        \item Engine mode will display the best moves as determined by a chess engine for the position
        \item Normal mode will disable the engine and beginner mode features. This is intended for regular play between experienced players
    \end{itemize}
\end{enumerate}}

\bigskip

\noindent{The following items are deemed to be \textbf{out of scope}:
\begin{enumerate}
    \item FIDE (International Chess Federation) standards for tournament appropriate chess equipment
    \item Tracking and support for alternate chess variants such as Chess960, Atomic Chess, King of the Hill. More information found here: \cite{ListOfChessVariants2022}.
    \item Proper tracking of alternate starting positions like puzzles
    \item Proper tracking of illegal moves and rule violations when warnings are ignored
\end{enumerate}}

\section{Project Overview}
\subsection{System Context Diagram}
\subsection{Normal Operation}
\subsubsection{Description}
\subsubsection{Use Cases/Scenarios}

\subsection{Behaviour Overview}
\subsection{Undesired Scenario Handling}

\section{System Level Variables}
\subsection{Constants}

\begin{table}[H]
  \centering
      \setlength{\leftmargini}{0.4cm}
      \begin{tabular}{| >{\centering\arraybackslash}m{5cm} | 
        >{\centering\arraybackslash}m{2cm} | 
        >{\centering\arraybackslash}m{5cm} |}
      \hline
      \rowcolor[gray]{0.9}
      Constant & Unit & Value\\
      \hline
      Chess board width & inches & 12\\
     \hline
     Chess board length & inches & 12\\
     \hline
     Chess board tile width & inches & 1.5\\
     \hline 
     Chess board tile length & inches & 1.5\\ 
     \hline 
     Supply Power to Board & Volts & 110 VAC\\
     \hline
      \end{tabular}
  \label{Table}
  \end{table}

\subsection{Monitored Variables}

\begin{table}[H]
  \centering
      \setlength{\leftmargini}{0.4cm}
      \begin{tabular}{| >{\centering\arraybackslash}m{2.5cm} | 
        >{\centering\arraybackslash}m{2cm} | 
        >{\centering\arraybackslash}m{9cm} |}
      \hline
      \rowcolor[gray]{0.9}
      Variable & Units & Description\\
      \hline
      s\_a\{1-8\} & Volts & States of tiles a1 - a8 on the board. They are analog signals 
      converted to digital and the state of the tile is determined. The possible states of 
      each tile is empty, black/white pawn, black/white rook, black/white knight, 
      black/white bishop, black/white queen, black/white king. \\
      \hline
      s\_b\{1-8\} & Volts & States of tiles b1 - b8 on the board. " " \\
      \hline
      s\_c\{1-8\} & Volts & States git of tiles c1 - c8 on the board. " " \\
      \hline
      s\_d\{1-8\} & Volts & States of tiles d1 - d8 on the board. " " \\
      \hline
      s\_e\{1-8\} & Volts & States of tiles e1 - e8 on the board. " " \\
      \hline
      s\_f\{1-8\} & Volts & States of tiles f1 - f8 on the board. " " \\
      \hline
      s\_g\{1-8\} & Volts & States of tiles g1 - g8 on the board. " " \\
      \hline
      sw\_3pos\_p{1-2} & Volts & The three-position switch for both players is located
      on top of the board on their respective sides. It toggles between the beginner 
      advice, engine advice and no advice modes for each player.\\
      \hline
      tieB\_p\{1-2\} & Volts & The "draw" push-button for each player is located on 
      the top of the board on their respective sides. When both players press their button
      the game is a draw. \\
      \hline
      engine\_move & chess notation & The chess engine API provides best moves into 
      the system. \\
      \hline 
      \end{tabular}
  \label{Table}
  \end{table}

\subsection{Controlled Variables}

\begin{table}[H]
  \centering
      \setlength{\leftmargini}{0.4cm}
      \begin{tabular}{| >{\centering\arraybackslash}m{3cm} | 
        >{\centering\arraybackslash}m{2cm} | 
        >{\centering\arraybackslash}m{9cm} |}
      \hline
      \rowcolor[gray]{0.9}
      Variable & Units & Description\\
      \hline 
      LED\_row\{1-9\} & Volts & A total of 81 LEDS will be located under the board. They 
      are on the corner of each tile and illuminate based on conditions of the inputs. \\
      \hline 
      LCD\_Display & Volts & An LCD Display is located on the chess board to indicate
      best moves delivered by the engine. \\
      \hline 
      \end{tabular}
  \label{Table}
  \end{table}

\section{Requirements}
\subsection{Functional Requirements}
\subsection{Nonfunctional Requirements}

\newcounter{vnvSectionNfr}
\setcounter{vnvSectionNfr}{1}

\newcounter{nfrNum}
\setcounter{nfrNum}{1}

\subsubsection{Look and Feel Requirements}
\label{NFR_LF}
\paragraph{Appearance Requirements}
\begin{enumerate}[{LF}1., leftmargin=2\parindent]
    \item The product shall use white, black, grey, and brown as its primary colours.
    \item The product shall use green, red, and blue as its secondary colours.
\end{enumerate}

\paragraph{Style Requirements}
\begin{enumerate}[{LF}1., leftmargin=2\parindent, resume]
    \item The product shall look and feel similar enough to traditional chess boards and chess pieces that the target audience will 
    recognize the product as a chess set when encountering it for the first time. The level and speed of audience recognition achieved 
    by the design shall be described following the procedure given in Section 5.2.\thevnvSectionNfr\stepcounter{vnvSectionNfr}~of the VnV 
    (Verification and Validation) Plan.
\end{enumerate}



\subsubsection{Usability and Humanity Requirements}
\label{NFR_UH}
\paragraph{Ease of Use Requirements}
\begin{enumerate}[{UH}1., leftmargin=2\parindent]
    \item The system shall require the user to place chess pieces fully on their intended squares.
    \item Physical hardware components of the system will not impede the user during play.
\end{enumerate}

\paragraph{Personalization and Internationalization Requirements}
\begin{enumerate}[{UH}1., leftmargin=2\parindent, resume]
    \item The system will only display information in English.
    \item The system will only use the Arabic numerals.
\end{enumerate}

\paragraph{Learning Requirements}
\begin{enumerate}[{UH}1., leftmargin=2\parindent, resume]
    \item The product shall be able to be used by members of the public over with no previous training. Details on the learnability 
    of the system shall be described following the procedure given in Section 5.2.\thevnvSectionNfr\stepcounter{vnvSectionNfr}
    of the VnV Plan.
\end{enumerate}

\paragraph{Understandability and Politeness Requirements}
\begin{enumerate}[{UH}1., leftmargin=2\parindent, resume]
    \item All symbols and words shall be similar to historically used Chess symbols. \cite{ChessHistory2003}
\end{enumerate}

\paragraph{Accessibility Requirements}
\begin{enumerate}[{UH}1., leftmargin=2\parindent, resume]
    \item The system shall follow guidelines for correct size and colour contrast ratio for text to the background as stated in the \cite{WCAG2018}.
\end{enumerate}



\subsubsection{Performance Requirements}
\label{NFR_PR}
\paragraph{Speed and Latency Requirements}
\begin{enumerate}[{PR}1., leftmargin=2\parindent]
    \item The average time between a user placing down a piece and the visual model response shall be small.
    \item The maximum time between a user placing down a piece and the visual model response shall be small.
    \item The average time between a user picking up a piece and the visual board indicator response shall be small.
    \item The maximum time between a user picking up a piece and the visual board indicator response shall be small. 
    The degree of speed for PR1 through PR4 shall be described following the procedure given in Section 5.2.\thevnvSectionNfr\stepcounter{vnvSectionNfr}
    of the VnV Plan.
\end{enumerate}

%Modified from Safety-Critical Requirements to better cover the SRS rubric
\paragraph{Health and Safety-Critical Requirements}
\begin{enumerate}[{PR}1., leftmargin=2\parindent, resume]
    \item The system shall be properly grounded according to the Canadian Electrical Code. \cite{CanadianElectricalCode2021}
    \item The maximum power on any single wire shall be within the safety limits described in the Canadian Electrical Code.
\end{enumerate}

\paragraph{Precision or Accuracy Requirements}
\begin{enumerate}[{PR}1., leftmargin=2\parindent, resume]
    \item The software application game state will model the game state on the \progname{} hardware with a high degree of accuracy. 
    The level of accuracy shall be described following the procedure given in Section 5.2.\thevnvSectionNfr\stepcounter{vnvSectionNfr}
    of the VnV Plan.
        
\end{enumerate}

\paragraph{Reliability and Availability Requirements}
\begin{enumerate}[{PR}1., leftmargin=2\parindent, resume]
    \item The product shall be available with a high degree of uptime. The level of availability shall be described following the procedure 
    given in Section 5.2.\thevnvSectionNfr\stepcounter{vnvSectionNfr} of the VnV Plan.
\end{enumerate}

\paragraph{Robustness or Fault-Tolerance Requirements}
\begin{enumerate}[{PR}1., leftmargin=2\parindent, resume]
    \item The software application shall maintain the game state if the connection between the software and hardware systems is interrupted.
\end{enumerate}

\paragraph{Capacity Requirements}
\begin{enumerate}[{PR}1., leftmargin=2\parindent, resume]
    \item The software shall require computer memory to function effectively. The level of memory capacity required shall be described following
    the procedure given in Section 5.2.\thevnvSectionNfr\stepcounter{vnvSectionNfr} of the VnV Plan.
\end{enumerate}

\paragraph{Scalability or Extensibility Requirements}
\begin{enumerate}[{PR}1., leftmargin=2\parindent, resume]
    \item The product must support the addition of new features and components.
\end{enumerate}

\paragraph{Longevity Requirements}
\begin{enumerate}[{PR}1., leftmargin=2\parindent, resume]
    \item The product must be supported while the application remains deployed.
    \item The product will depend on the continued support of packages and libraries.
\end{enumerate}



\subsubsection{Operational and Environmental Requirements}
\label{NFR_OE}
\paragraph{Expected Physical Environment}
\begin{enumerate}[{OE}1., leftmargin=2\parindent]
    \item The hardware and software systems shall be close enough to each other to facilitate communication. The degree of proximity required 
    shall be described following the procedure given in Section 5.2.\thevnvSectionNfr\stepcounter{vnvSectionNfr} of the VnV Plan.
    \item The area shall be clear of potentially dangerous or harmful environmental factors.
\end{enumerate}

\paragraph{Requirements for Interfacing with Adjacent Systems}
\begin{enumerate}[{OE}1., leftmargin=2\parindent, resume]
    \item The system shall interface with an external server to make requests to a chess engine.
\end{enumerate}

\paragraph{Productization Requirements}
\begin{enumerate}[{OE}1., leftmargin=2\parindent, resume]
    \item The product shall be deployed to a public website where users may access it.
\end{enumerate}

\paragraph{Release Requirements}
\begin{enumerate}[{OE}1., leftmargin=2\parindent, resume]
    \item The product will be tested for bugs and issues. These issues will be fixed and the application will be redeployed accordingly.
\end{enumerate}



\subsubsection{Maintainability and Support Requirements}
\label{NFR_MS}
\paragraph{Maintenance Requirements}
\begin{enumerate}[{MS}1., leftmargin=2\parindent]
    \item The product shall be maintained actively by the developers until the \progname{} team graduates.
\end{enumerate}

\paragraph{Supportability Requirements}
N/A

\paragraph{Adaptability Requirements}
\begin{enumerate}[{MS}1., leftmargin=2\parindent, resume]
    \item The software application will be able to be hosted on Apple, Windows, and Linux devices.
    \item The product shall be accessible from any web browser.
\end{enumerate}



\subsubsection{Security Requirements}
\label{NFR_SR}
\paragraph{Access Requirements}
\begin{enumerate}[{SR}1., leftmargin=2\parindent]
    \item Only the \progname{} team are able to modify the software system.
\end{enumerate}

\paragraph{Integrity Requirements}
\begin{enumerate}[{SR}1., leftmargin=2\parindent, resume]
    \item The product will not store game data after a game has concluded.
\end{enumerate}

\paragraph{Privacy Requirements}
\begin{enumerate}[{SR}1., leftmargin=2\parindent, resume]
    \item The product will not store or collect user data.
\end{enumerate}

\paragraph{Audit Requirements}
\begin{enumerate}[{SR}1., leftmargin=2\parindent, resume]
    \item Requirements shall be easy to follow and verify against both the system and the VnV plan in order to facilitate regular inspections.
\end{enumerate}

\paragraph{Immunity Requirements}
N/A



\subsubsection{Political and Cultural Requirements}
\label{NFR_PC}
\paragraph{Cultural Requirements}
\begin{enumerate}[{PC}1., leftmargin=2\parindent]
    \item The product will not use and terms or symbols that are deemed offensive to any culture.
\end{enumerate}

\paragraph{Political Requirements}
N/A



\subsubsection{Legal Requirements}
\label{NFR_Legal}
\paragraph{Compliance Requirements}
\begin{enumerate}[{LR}1., leftmargin=2\parindent]
    \item The system shall comply with the Canadian Electrical Code \cite{CanadianElectricalCode2021}.
\end{enumerate}

\paragraph{Standards Requirements}
\begin{enumerate}[{LR}1., leftmargin=2\parindent, resume]
    \item The product shall follow \cite{WCAG2018}.
\end{enumerate}


\section{Likely Changes}
\section{Unlikely Changes}

\section{Traceability Matrix}

\appendix
\section{Values of Auxiliary Constants}

\newpage

\appendix
\section{Reflection}
\subsection{Skills for Success}
\subsection{Knowledge and Learning Approaches}

\newpage

\bibliographystyle {plainnat}
\bibliography {../../refs/References}
\end{document}